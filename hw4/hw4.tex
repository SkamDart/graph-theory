% !TeX spellcheck = en_US

% ===============================================
% MATH 412: Graph Theory         Fall 2017
% hw_revised.tex




\documentclass{article}

\usepackage[margin=1in]{geometry} 
\usepackage{amsmath,amsthm,amssymb, enumitem, mathtools}

\newcommand{\R}{\mathbb{R}}  
\newcommand{\Z}{\mathbb{Z}}
\newcommand{\N}{\mathbb{N}}
\newcommand{\Q}{\mathbb{Q}}
\newcommand{\C}{\mathbb{C}}
\newcommand{\K}[2]{K_{\floor*{#1} \ceil*{#2}}}

\newenvironment{theorem}[2][Theorem]{\begin{trivlist}
\item[\hskip \labelsep {\bfseries #1}\hskip \labelsep {\bfseries #2.}]}{\end{trivlist}}
\newenvironment{lemma}[2][Lemma]{\begin{trivlist}
\item[\hskip \labelsep {\bfseries #1}\hskip \labelsep {\bfseries #2.}]}{\end{trivlist}}
\newenvironment{exercise}[2][Exercise]{\begin{trivlist}
\item[\hskip \labelsep {\bfseries #1}\hskip \labelsep {\bfseries #2.}]}{\end{trivlist}}
\newenvironment{problem}[2][Problem]{\begin{trivlist}
\item[\hskip \labelsep {\bfseries #1}\hskip \labelsep {\bfseries #2.}]}{\end{trivlist}}
\newenvironment{question}[2][Question]{\begin{trivlist}
\item[\hskip \labelsep {\bfseries #1}\hskip \labelsep {\bfseries #2.}]}{\end{trivlist}}
\newenvironment{corollary}[2][Corollary]{\begin{trivlist}
\item[\hskip \labelsep {\bfseries #1}\hskip \labelsep {\bfseries #2.}]}{\end{trivlist}}

\DeclarePairedDelimiter\ceil{\lceil}{\rceil}
\DeclarePairedDelimiter\floor{\lfloor}{\rfloor}

\newenvironment{solution}{\begin{proof}[Solution]}{\end{proof}}
\newenvironment{claim}[2][Claim]{\begin{trivlist}
		\item[\hskip \labelsep {\bfseries #1}\hskip \labelsep {\bfseries #2}]}{\end{trivlist}}

\begin{document}
\title{Homework 4} 
\author{Cameron Dart\\ Graph Theory} 

\maketitle

\noindent \textbf{Problem 1}
\begin{claim}{}
	For all odd $n \geq 3$ there exists an $n-$vertex tournament $T$ such that every vertex in $T$ is a king. There does not exist a tournament with $4$ vertices that has this property.
\end{claim}
\begin{proof}
	Suppose our graph $T$ has $n$ vertices. Put these $n$ vertices on a circle. For every vertex $u \in V(T)$ draw an edge from $u$ to the next $\frac{n-1}{2}$ vertices on the circle. Repeat this for all $n$ vertices. As a result, $T$ has $\frac{n(n - 1)}{2}$ edges and the underlying graph of $T$ is $K_n$. So $T$ must be a tournament. Suppose $u, v \in V(T)$. It must be the case that either $u \rightarrow v$ or there exists a $w \in N(u)$ so that $w \rightarrow v$. So by definition every vertex in $T$ is a king.\\ \\
	Now assume there exists $4$ vertex tournament $T'$ where every vertex is a king. Construct $T'$ in the same method as above. It follows that $T'$ has $8$ edges and we reach a contradiction. There does not exist a $4$ vertex tournament such that every vertex is a king.
\end{proof}

\noindent \textbf{Problem 2}
\begin{claim}{1}
	Suppose $G$ is a simple, connected graph with at least $3$ vertices. Then the following claims are equivalent.
	\begin{enumerate}[label=\alph*)]
		\item $G$ is a tree
		\item Every induced subgraph of $G$ has a vertex of degree at most $1$
		\item For every vertex $v$ in $G$, the number of components of $G - v$ equals the degree of $v$
	\end{enumerate}
\end{claim}
\begin{proof}
	$a \implies b$) Suppose $G$ is a tree. $G$ does not contain any cycles by definition so any induced subgraph of $G$ must contain a leaf. Therefore, it has a vertex of degree $1$. Now consider an induced subgraph of $G$ that is not connected. Clearly, any isolated vertex will have degree zero. Hence, it is true that any induced subgraph of $G$ contains a vertex of degree at most $1$ and $a \implies b$.\\ \\
	$b \implies a)$ Consider the contrapositive of the statement. If $G$ is not a tree, then there exists an induced subgraph of $G$ that has a all vertices degree of greater than $1$. Since $G$ is not a tree, there exists a cycle. Consider the induced subgraph that is a cycle in $G$. Clearly, every vertex in the induced cycle has degree greater than one. So it is true that $b \implies a$\\ \\
	$c \implies a)$ Consider the contrapositive of $c \implies a$. If $G$ is not a tree, then there exists a vertex $v \in V(G)$ such that the number of components of $G - v$ is not the degree of $v$. Since $G$ is not a tree it has a cycle. Pick any vertex $v$ in the cycle contained. $d_G(v)$ is at least two. If you delete $v$, then there still exists $1$ component since the cycle is now a path.
\end{proof}
\newpage
\noindent \textbf{Problem 3}
\begin{claim}{}
	Suppose $G$ has vertex set $\left \{ v_1, v_2, ..., v_n \right \}$ in which $n-1$ vertices $v_1, ... , v_{n-1}$ have the property such that $G - v_i$ for $i \in 1,2,..., n - 1$ is a tree. $G$ must either be a $P_n$ or $S_n$ 
\end{claim}
\begin{proof}
	First, let $G$ be a cycle of length $n$. Let $G' = G - v$ be the graph that deletes an arbitrary $v \in V(G)$. It is easy to see that $G'$ is a path of length $n - 1$ which is connected and acyclic. Therefore $G'$ is a tree.\\ \\
	Now suppose that $G$ is $S_n$ and $s$ is the vertex of degree $n - 1$ in $S_n$. Choose our vertex set to be the independent set of size $n - 1$ contained in $S_n$. Let $G' = G - v$ where $v$ is in the independent set of $G$. $G'$ must be connected since every vertex in our independent set is still adjacent $s$. Assume there is a cycle in $G'$. $G'$ is bipartite so there cannot exist an odd cycle. If there was a cycle, it would be even and we would need to traverse two elements from our two independent sets, however, one of our independent sets contains just $s$.	Hence, we reach a contradiction and there cannot exist a cycle in $G'$. It follows that $G'$ is a tree because it is connected and acyclic.\\ \\
	Assume there exists another graph $G$ with the properties above and seek contradiction. 
	If a graph $G$ has $n$ vertices and $n$ edges, then the degree sum is equal to $2n$. So if there exist multiple vertices with degree greater than $2$, then there must be corresponding vertices with degree $0$. Which contradicts  o because we can choose a vertex set that contains isolated vertices. And it is not connected therefore not a tree.
\end{proof}

\noindent \textbf{Problem 4}
\begin{claim}{}
	For every $n \geq 2$ the minimum Wiener index of a $n-$vertex simple graph $G$ with $n$ edges is \\$2(n-1)^2 - 2$. The max value of $D(G)$ for a $5-$vertex tree is $40$
\end{claim}
\begin{proof}
	The minimal Wiener index for $n-$vertex $n-1$ edge graph $G$ is a star. The wiener index of a star is $(n-1)^2$. Adding a single edge to this star graph creates exactly one cycle and has the minimal Wiener index for a graph with $n$ edges and $n$ vertices. Namely, there are $(n-1)^2 - 1$ edges. Since we shortened the distance between the two vertices in our cycle that aren't the center of the star.\\
	We have a maximal wiener index for a $n-$vertex graph when we consider $P_n$. Hence, $P_5$ has a maximal wiener index of $2(4) + 4(3) + 6(2) + 8(1) = 40$ by inspection.
\end{proof}

\noindent \textbf{Problem 5}
\begin{claim}{}
	There are $\frac{(n - 2)!}{5!}(n - 7)!$ prufer codes with one vertex of degree $6$ and $6$ leaves
\end{claim}
\begin{proof}
	The center appears $5$ times in our prufer code so there are $\binom{n-2}{5}$ ways to place it in our prufer and $(n-7)!$ ways to order the rest of our vertices. It follows there are $\binom{n-2}{5}(n-7)!$ prufer trees with the given criteria.
\end{proof}

\end{document}