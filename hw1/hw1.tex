
% ===============================================
% MATH 412: Graph Theory         Fall 2017
% hw_revised.tex
% Template for revised homework submission
% ===============================================
%         READ THE FOLLOWING CAREFULLY!!!
% ===============================================
% When you produce a PDF version of this document
% to turn in, change the filename to hwX-name.pdf
% replacing X with the homework assignment number
% and name with your last name.
% ===============================================


% -----------------------------------------------
% The preamble that follows can be ignored. Go on
% down to the section that says "START HERE" 
% -----------------------------------------------

\documentclass{article}

\usepackage[margin=1in]{geometry} 
\usepackage{amsmath,amsthm,amssymb, enumitem}

\newcommand{\R}{\mathbb{R}}  
\newcommand{\Z}{\mathbb{Z}}
\newcommand{\N}{\mathbb{N}}
\newcommand{\Q}{\mathbb{Q}}
\newcommand{\C}{\mathbb{C}}

\newenvironment{theorem}[2][Theorem]{\begin{trivlist}
\item[\hskip \labelsep {\bfseries #1}\hskip \labelsep {\bfseries #2.}]}{\end{trivlist}}
\newenvironment{lemma}[2][Lemma]{\begin{trivlist}
\item[\hskip \labelsep {\bfseries #1}\hskip \labelsep {\bfseries #2.}]}{\end{trivlist}}
\newenvironment{exercise}[2][Exercise]{\begin{trivlist}
\item[\hskip \labelsep {\bfseries #1}\hskip \labelsep {\bfseries #2.}]}{\end{trivlist}}
\newenvironment{problem}[2][Problem]{\begin{trivlist}
\item[\hskip \labelsep {\bfseries #1}\hskip \labelsep {\bfseries #2.}]}{\end{trivlist}}
\newenvironment{question}[2][Question]{\begin{trivlist}
\item[\hskip \labelsep {\bfseries #1}\hskip \labelsep {\bfseries #2.}]}{\end{trivlist}}
\newenvironment{corollary}[2][Corollary]{\begin{trivlist}
\item[\hskip \labelsep {\bfseries #1}\hskip \labelsep {\bfseries #2.}]}{\end{trivlist}}

\newenvironment{solution}{\begin{proof}[Solution]}{\end{proof}}
\newenvironment{claim}[2][Claim]{\begin{trivlist}
		\item[\hskip \labelsep {\bfseries #1}\hskip \labelsep {\bfseries #2}]}{\end{trivlist}}

\begin{document}
\title{Homework 1} % Change homework number
\author{Cameron Dart\\ Graph Theory} % Replace "Author's Name" with your name

\maketitle

%%%%%%%%%%%%%%%%%%%%%%%%%%%%%%%%%%%%%%%%
\noindent\textbf{Problem 1}
Suppose $A = \{ K_n \}, B = \{ \bar C_n \} , C = \{P_n\}, D = \{ \textit{bipartite graphs} \} $
\begin{claim}{}
$A \cap B = \emptyset$
\end{claim}
\begin{proof}
Assume $A \cap B \neq \emptyset$ and seek contradiction. Since the complement of a multigraph is undefined assume $n > 2$. $K_n$ has
$\frac{n(n-1)}{2}$ edges and $\bar C_n$ has $\frac{n (n-1)}{2} - n$ edges. Obviously, $\frac{n (n-1)}{2} > \frac{n(n-1)}{2} - n$ for $n > 2$. Hence, no graphs from either family are isomorphic and we reach our sought contradiction.
\end{proof}

\begin{claim}{}
$A \cap C = \{ K_1, K_2 \}$
\end{claim}
\begin{proof}
Clearly, $K_1 \cong P_1$ and $K_2 \cong P_2$. Suppose $n > 2$. $P_n$ has $n - 1$ edges while $K_n$ has $\frac{n(n-1)}{2}$ edges and $n-1 \neq \frac{n(n-1)}{2}$ for $n > 2$. Hence, $A \cap C$ must only contain $K_1, K_2$.
\end{proof}

\begin{claim}{} 
$A \cap D = \{ K_1, K_2 \}$
\end{claim}
\begin{proof}
$K_1$ and $K_2$ do not contain any odd cycles so they must be bipartite.\\
For any $K_n$ where $n > 2$, we need $n$ independent sets to cover $V(K_n)$ since every vertex in $K_n$ is adjacent. So $K_n$ is not bipartite for $n > 2$. 
\end{proof}

\begin{claim}{}
$B \cap C = \emptyset$
\end{claim}
\begin{proof}
Assume $B \cap C \neq \emptyset$ and seek contradiction.
If two graphs are isomorphic, then by definition their complements are isomorphic. $|E(\bar C_n)| = \frac{n(n-1)}{2} - n $ and $|E(P_n)| = n - 1$. Clearly for $n > 2$, this is not true. Hence, we reach a contradiction.
\end{proof}

\begin{claim}{}
$B \cap D = \{ \bar C_3, \bar C_4 \}$
\end{claim}
\begin{proof}
Consider the complement of $C_3$, a graph containing $3$ vertices and no edges. We can easily form two independent sets since there are no edges in our graph. So $\bar C_3$ is bipartite.\\ \\
Now consider $\bar C_4$, a graph containing two independent paths or two $P_2$ graphs. We can form two independent sets with one vertex from each independent graph which implies that $\bar C_4$ is bipartite. \\ \\
Suppose $n = 5$. $\bar C_5$ has a cycle of length $5$ so it cannot be bipartite. \\ \\
Now consider $\bar C_n$ for $n > 5$. Label each vertex in $C_n$ by defining a bijection from the first $n$ natural numbers to the vertices in our graph $\bar C_n$. Then for any $n$ we can create a cycle of length $3$ between the vertices in $\bar C_n$ labeled $1,3,5$. Hence, $\bar C_n$ is not bipartite since there exists an odd cycle.

\end{proof}
\newpage
\begin{claim}{}
$C \cap D = \{ P_n \}$
\end{claim}
\begin{proof}
Suppose $n = 1$. Clearly, the singleton graph is bipartite. Namely, we form two sets, $S = \emptyset, T = \{ \textit{the only vertex in } P_n\}$. $S \cup T = V(P_1)$ and $S \cap T = \emptyset$. So by definition $P_1$ is bipartite. \\ \\
Assume for $1 \leq n \leq k$ that the graph $P_n$ is bipartite. \\ \\
Consider $P_{n + 1}$. Note, the parity of $n$ is trivial. We define a bijection from the first $n + 1$ natural numbers to the $n + 1$ vertices in $P_{n + 1}$ effectively labeling each vertex. Then we can construct two sets $X = \{ v_\textit{even} \}, Y = \{ v_\textit{odd} \}$. It follows, $X \cap Y = \emptyset$ and $X \cup Y = V(P_{n + 1})$. Thus, we have created two independent sets from the vertices of $P_{n + 1}$ which implies that $P_{n+1}$ is bipartite and $P_n$ is bipartite for any $n$. 
\end{proof}
%%%%%%%%%%%%%%%%%%%%%%%%%%%%%%%%%%%%%%%%%%%%%%%%%%%%%%%%%%%%%%%%%%%%%%%%%%%%%%%%%%%%%%%%%%%%%%%%%%%%%%%%%%%%%%%%%%%%%%%%%%%%%%%%%%%%%%%%%%%%%%%%%%%%%%%%%%%%%%%%%%%%%%%%%%%%%%%%%%%%%%%%%%%%%%%%%%%%%%%%%%%%%%%%%%%%%%%%%%%%%%%%%%%%%%%%%%%%%%%%%%%%%%%%%%%%%%%%%%%%%%%%%%%%%%%%%%%%%%%%%%%%%%%%%%%%%%%%%%%%%%%%%%%%%%%%%%%%%%%%%%%%%%%%%%%%%%%%%%%%%%%%%%%%%%%%%%%%%%%%%%%%%%%%%%%%%%%%%%%%%%%%%%%%%%%%%%%%%%%%%%%%%%%%%%%%%%%%%%%%%%%%%%%%%%%%%%%%%%%%%%%%%%%%%%%%%%%%%%%%%%%%%%%%%%%%%%%%%%%%%%%%%%%%%%%%%%%%%%%%%%%%%%%%%%%%%%%%%%%%%%%%%%%%%%%%%%%%%%%%%%%%%%%%%%%%%%%%%%%%%%%%%%%%%%%%%%%%%%%%%%%%%%%%%%%%%%%
%%%%%%%%%%%%%%%%%%%%%%%%%%%%%%%%%%%%%%%%%%%%%%%%%%%%%%%%%%%%%%%%%%%%%%%%%%%%%%%%%%%%%%%%%%%%%%%%%%%%%%%%%%%%%%%%%%%%
%%%%%%%%%%%%%%%%%%%%%%%%%%%%%%%%%%%%%%%%%%%%%%%%%%%%%%%%%%%%%%%%%%%%%%%%%%%%%%%%%%%%%%%%%%%%%%%%%%%%%%%%%%%%%%%%%%%%
\noindent\textbf{Problem 2}
\begin{claim}{}
$G_1 \not \cong G_2, G_1 \not \cong G_3, G_2 \cong G_3$
\end{claim}
\begin{proof}
While $G_2, G_3$ contain cycles of length $5$, $G_1$ does not. So $G_2, G_3$ are not isomorphic to $G_1$. \\
Now define a bijection $f$ as follows:
\begin{align*}
f : V(G_2) &\to V(G_3)\\
v_{1}  \mapsto s_{1},    \,  & v_{11} \mapsto s_{11}  \\
v_{3}  \mapsto s_{3},    \,  & v_{12} \mapsto s_{12}  \\
v_{5}  \mapsto s_{5},    \,  & v_{13} \mapsto s_{13}  \\
v_{7}  \mapsto s_{7},    \,  & v_{14} \mapsto s_{14}  \\
v_{9}  \mapsto s_{9},    \,  & v_{15} \mapsto s_{15}  \\
v_{2}  \mapsto s_{2},    \,  & v_{16} \mapsto s_{16}  \\
v_{4}  \mapsto s_{4},    \,  & v_{17} \mapsto s_{17}  \\
v_{6}  \mapsto s_{6},    \,  & v_{18} \mapsto s_{18}  \\
v_{8}  \mapsto s_{8},    \,  & v_{19} \mapsto s_{19}  \\
v_{10} \mapsto s_{10},   \,  & v_{20} \mapsto s_{20}  \\
\end{align*}
Hence, our graphs are isomorphic. In fact, $G_2, G_3$ are planar representations of a pentahedron.
\textit{Please see attached paper at the end for drawings of graph.}
\end{proof}
%
\newpage
%%%%%%%%%%%%%%%%%%%%%%%%%%%%%%%%%%%%%%%%%%%%%%%%%%%%%%%%%%%%%%%%%%%%%%%%%%%%%%%%%%%%%%%%%%%%%%%%%%%%%%%%%%%%%%%%%%%%%%%%%%%%%%%%%%%%%%%%%%%%%%%%%%%%%%%%%%%%%%%%%%%%%%%%%%%%%%%%%%%%%%%%%%%%%%%%%%%%%%%%%%%%%%%%%%%%%%%%%%%%%%%%%%%%%%%%%%%%%%%%%%%%%%%%%%%%%%%%%%%%%%%%%%%%%%%%%%%%%%%%%%%%%%%%%%%%%%%%%%%%%%%%%%%%%%%%%%%%%%%%%%%%%%%%%%%%%%%%%%%%%%%%%%%%%%%%%%%%%%%%%%%%%%%%%%%%%%%%%%%%%%%%%%%%%%%%%%%%%%%%%%%%%%%%%%%%%%%%%%%%%%%%%%%%%%%%%%%%%%%%%%%%%%%%%%%%%%%%%%%%%%%%%%%%%%%%%%%%%%%%%%%%%%%%%%%%%%%%%%%%%%%%%%%%%%%%%%%%%%%%%%%%%%%%%%%%%%%%%%%%%%%%%%%%%%%%%%%%%%%%%%%%%%%%%%%%%%%%%%%%%%%%%%%%%%%%%%%%
%%%%%%%%%%%%%%%%%%%%%%%%%%%%%%%%%%%%%%%%%%%%%%%%%%%%%%%%%%%%%%%%%%%%%%%%%%%%%%%%%%%%%%%%%%%%%%%%%%%%%%%%%%%%%%%%%%%%
%%%%%%%%%%%%%%%%%%%%%%%%%%%%%%%%%%%%%%%%%%%%%%%%%%%%%%%%%%%%%%%%%%%%%%%%%%%%%%%%%%%%%%%%%%%%%%%%%%%%%%%%%%%%%%%%%%%%
\noindent\textbf{Problem 3}\\
Let $n \geq 4$. Suppose $G$ is a simple graph with $n$ vertices.
\begin{claim}{}
If every $3$ vertex induced subgraph of $G$ has either $0$ or $1$ edges, then the degree of every vertex in $G$ is at most $1$.
\end{claim}
\begin{proof}
Suppose every $3$ vertex induced subgraph of $G$ has $0$ or $1$ edges. Consider an arbitrary vertex $v \in V(G)$. If the degree of $v$ is greater than $1$, we can take an induced subgraph of two vertices adjacent in $v$ and we have found an induced subgraph with at least two edges. This contradicts our original assumption that every induced subgraph of $G$ has $0,1$ edges. Thus, the degree of every vertex in $G$ must be less or equal to $1$. 
\end{proof}

\begin{claim}{}
If every $3$ vertex induced subgraph of $G$ has either $1$ or $2$ edges, then $G = \{ \bar C_4, C_4, C_5, P_4 \}$
\end{claim}
\begin{proof}
Any induced subgraph of $\bar C_4$ includes one of the $P_2$ graphs and a single node. Hence, it has $1$ edge.\\ \\
Any induced subgraph of $C_4$ contains a path $P_3$ or a $P_2$ and a singleton node. Thus, $C_4$ contains induced subgraphs with $1,2$ edges. \\ \\
The same logic as above holds for $C_5$ and $P_4$.\\ \\
Suppose $G$ is not one of the graphs described above. Choose arbitrary $u,v,w \in V(G)$.\
If 
\end{proof}

\begin{claim}{}
If every 3 vertex induced subgraph of $G$ has either $1$ or $3$ edges, then $G$ consists of two graphs $K_n$ or a single $K_n$ graph.
\end{claim}
\begin{proof}
Suppose $G$ is a simple graph such that every $3$ vertex induced subgraph of $G$ has either $1$ or $3$ edges. Fix three arbitrary vertices $u,v,w \in V(G)$. \\ \\
Consider the case where the induced subgraph of vertices $u,v,w$ contains one edge. Without the loss of generality assume that $u,v$ are adjacent. Then $w$ must not be adjacent to either $u$ or $v$. So we must have two connected components $H_1, H_2$ in our graph and $w$ is any arbitrary vertex in our component not containing $u,v$. Since $u,v$ are arbitrary all of our vertices in that connected component must be adjacent. Hence, it is a complete graph. \\ \\
Suppose the induced subgraph of vertices $u,v,w$ contains $3$ edges. Since these vertices are arbitrary all of our vertices must be adjacent. Hence, $u,v,w$ are included in the same complete graph $K_n$ where $n 
\geq 4$ and there could be at most one other componenet $m \geq 4$.
\\ \\ Thus we have shown that $G$ must be comprised of one or two complete graphs.
\end{proof}

%\begin{enumerate}[label=\alph*)]
%	\item If every 3 vertex induced subgraph of $G$ has either $0$ or $1$ edges, then the degree of any vertex contained in $G$ must be at most $1$. 
%    \item ****** If every 3 vertex induced subgraph of $G$ has either $1$ or $2$ edges, then there are no $3$ cycles in $G$ and \textit{there are no unconnected subgraphs}
%    \item ****** If every 3 vertex induced subgraph of $G$ has either $1$ or $3$ edges, then $G$ is either one complete graph or two complete graphs.
%\end{enumerate}
%
\newpage
%%%%%%%%%%%%%%%%%%%%%%%%%%%%%%%%%%%%%%%%%%%%%%%%%%%%%%%%%%%%%%%%%%%%%%%%%%%%%%%%%%%%%%%%%%%%%%%%%%%%%%%%%%%%%%%%%%%%%%%%%%%%%%%%%%%%%%%%%%%%%%%%%%%%%%%%%%%%%%%%%%%%%%%%%%%%%%%%%%%%%%%%%%%%%%%%%%%%%%%%%%%%%%%%%%%%%%%%%%%%%%%%%%%%%%%%%%%%%%%%%%%%%%%%%%%%%%%%%%%%%%%%%%%%%%%%%%%%%%%%%%%%%%%%%%%%%%%%%%%%%%%%%%%%%%%%%%%%%%%%%%%%%%%%%%%%%%%%%%%%%%%%%%%%%%%%%%%%%%%%%%%%%%%%%%%%%%%%%%%%%%%%%%%%%%%%%%%%%%%%%%%%%%%%%%%%%%%%%%%%%%%%%%%%%%%%%%%%%%%%%%%%%%%%%%%%%%%%%%%%%%%%%%%%%%%%%%%%%%%%%%%%%%%%%%%%%%%%%%%%%%%%%%%%%%%%%%%%%%%%%%%%%%%%%%%%%%%%%%%%%%%%%%%%%%%%%%%%%%%%%%%%%%%%%%%%%%%%%%%%%%%%%%%%%%%%%%%%
%%%%%%%%%%%%%%%%%%%%%%%%%%%%%%%%%%%%%%%%%%%%%%%%%%%%%%%%%%%%%%%%%%%%%%%%%%%%%%%%%%%%%%%%%%%%%%%%%%%%%%%%%%%%%%%%%%%%
%%%%%%%%%%%%%%%%%%%%%%%%%%%%%%%%%%%%%%%%%%%%%%%%%%%%%%%%%%%%%%%%%%%%%%%%%%%%%%%%%%%%%%%%%%%%%%%%%%%%%%%%%%%%%%%%%%%%
\noindent\textbf{Problem 4}\textit{ Please see attached sheet at end for this problem}\\
\noindent\textbf{Problem 5}\\
Suppose $G$ is a simple graph with incidence matrix $M$. \\
Then the $(i,i)$-th entry of $(MM^T)$ represents the order of vertex $i$. \\ 
Also, the $(i,j)$-th where $i \neq j$ of $(MM^T)$ represents whether or not two vertices $i,j$ are adjacent. \\
The $(i,j)$ entires of $(M^TM)$ represent whether edges $i,j$ share a common vertex.\\
$(M^TM)_{i,i} = 2$ for all $i$. See proof below.\\ \\
Note, $M^TM$ and $MM^T$ are always symmetric.\\
$(M^TM)^T = M^T (M^T)^T = M^TM$ and $(MM^T)^T = (M^T)^T M^T = MM^T$

\begin{claim}{}
	The diagonal entries of $M^T M = 2$.
\end{claim}

\begin{proof}
Suppose $M$ is an arbitrary $n \times m$ incidence matrix representation of a simple graph $G$ with $n$ nodes and $m$ edges. Since $G$ is simple, there is at most $1$ edge between any two vertices in $G$. Let the $j$-th column of $M$ be $M_j$. The $(i,j)$-th entry of $M^TM$ measures the dot product between $M_i$ and $M_j$. Hence, $(j,j)$-th entry computes $M^T_jM_j= ||M||^2$. It follows from the column properties of $M$ that the product $M^T_jM_j = 2$. The geometric interpretation of this is every edge has two vertices in common with itself.


\end{proof}
%%%%%%%%%%%%%%%%%%%%%%%
% Nothing to see here %
%%%%%%%%%%%%%%%%%%%%%%%
\end{document}
