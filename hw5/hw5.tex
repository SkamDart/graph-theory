% !TeX spellcheck = en_US

% ===============================================
% MATH 412: Graph Theory         Fall 2017
% hw_revised.tex




\documentclass{article}

\usepackage[margin=1in]{geometry} 
\usepackage{amsmath,amsthm,amssymb, enumitem, mathtools}

\newcommand{\R}{\mathbb{R}}  
\newcommand{\Z}{\mathbb{Z}}
\newcommand{\N}{\mathbb{N}}
\newcommand{\Q}{\mathbb{Q}}
\newcommand{\C}{\mathbb{C}}
\newcommand{\K}[2]{K_{\floor*{#1} \ceil*{#2}}}

\newenvironment{theorem}[2][Theorem]{\begin{trivlist}
\item[\hskip \labelsep {\bfseries #1}\hskip \labelsep {\bfseries #2.}]}{\end{trivlist}}
\newenvironment{lemma}[2][Lemma]{\begin{trivlist}
\item[\hskip \labelsep {\bfseries #1}\hskip \labelsep {\bfseries #2.}]}{\end{trivlist}}
\newenvironment{exercise}[2][Exercise]{\begin{trivlist}
\item[\hskip \labelsep {\bfseries #1}\hskip \labelsep {\bfseries #2.}]}{\end{trivlist}}
\newenvironment{problem}[2][Problem]{\begin{trivlist}
\item[\hskip \labelsep {\bfseries #1}\hskip \labelsep {\bfseries #2.}]}{\end{trivlist}}
\newenvironment{question}[2][Question]{\begin{trivlist}
\item[\hskip \labelsep {\bfseries #1}\hskip \labelsep {\bfseries #2.}]}{\end{trivlist}}
\newenvironment{corollary}[2][Corollary]{\begin{trivlist}
\item[\hskip \labelsep {\bfseries #1}\hskip \labelsep {\bfseries #2.}]}{\end{trivlist}}

\DeclarePairedDelimiter\ceil{\lceil}{\rceil}
\DeclarePairedDelimiter\floor{\lfloor}{\rfloor}

\newenvironment{solution}{\begin{proof}[Solution]}{\end{proof}}
\newenvironment{claim}[2][Claim]{\begin{trivlist}
		\item[\hskip \labelsep {\bfseries #1}\hskip \labelsep {\bfseries #2}]}{\end{trivlist}}

\begin{document}
\title{Homework 4} 
\author{Cameron Dart\\ Graph Theory} 

\maketitle

\noindent \textbf{Problem 1}
\begin{claim}{}
	In the graph provided, we can get minimum spanning trees from Kruskal's and Prim's Algorithm.
\end{claim}
\begin{proof}
Let $xy$ denote the edge connecting the two vertices $x$ and $y$. In Prim's, we always keep a connected component, starting with a single vertex and adding the next adjacent vertex with the smallest edge weight. Consider Prim's algorithm startin from vertex $a$.
\\\\\textbf{Prim's Algorithm:} ae, ef, fb, bc, cg, gh, hd, dp, pl, lk, ko, on, nj, ji, im
\\\\ In Kruskal's, we do not keep a connected componented. Instead, at each stage, we look globally at the smallest edge and take it if it does not create a cycle in the current forest.
\\\\\textbf{Kruskal's Algorithm:} ae, bf, cg, dh, ad, am, mp, ef, pl, bc, ij, kl, ko, jn\\
\end{proof}

\noindent \textbf{Problem 2}
\begin{claim}{}
	$\tau(K_{2,4}) = $ and $K_{2,4}$ has $2$ nonisomorphic spanning trees
\end{claim}
\begin{proof}
Let $Q$ be the Laplacian Matrix for $K_{2,4}$ 
\[
\begin{bmatrix}
	4  &  0  & -1 & -1 & -1 & -1 \\
	0  &  4  & -1 & -1 & -1 & -1 \\
	-1 & -1  &  2 &  0 &  0 &  0 \\
	-1 & -1  &  0 &  2 &  0 &  0 \\
	-1 & -1  &  0 &  0 &  2 &  0 \\
	-1 & -1  &  0 &  0 &  0 &  2 
\end{bmatrix}
\]
Suppose $Q^*$ is the matrix resulting from deleting the $1st$ row and $1st$ column of $Q$
\[
\begin{bmatrix}
 4  & -1 & -1 & -1 & -1 \\
-1  &  2 &  0 &  0 &  0 \\
-1  &  0 &  2 &  0 &  0 \\
-1  &  0 &  0 &  2 &  0 \\
-1  &  0 &  0 &  0 &  2 
\end{bmatrix}
\]
\end{proof}
By the \textbf{Matrix Tree Theorem} $\tau(K_{2,4}) = (-1)^(2) \det(Q^*)$
$\det(Q^*) = 32$
Hence, $\tau(K_{2, 4}) = \det(Q^*) = 32$. So $K_{2,4}$ has $32$ spanning trees. \\
However, there are only two up to isomorphism. Specifically, 
\\\noindent \textbf{Problem 3}
\begin{claim}{}
Let $G = (X,Y ; E)$ be a bipartite simple graph with partite sets $X$ and $Y$, where $|X| = |Y| = n$. Then $\alpha'(G) \geq \min{, 2 \delta(G)}$ 
\end{claim}
\begin{proof}

\end{proof}

\noindent \textbf{Problem 4}
\begin{claim}{}
If $A$ is a $0, 1$ matrix so that a \textit{line} in A is a row or column and two $1$'s in $A$ are independent if no line contains both of them. Then the max number of pairwise independent $1's$ is equal to the minimum number of lines covering all $1's$ in $A$.
\end{claim}
\begin{proof}

\end{proof}

\noindent \textbf{Problem 5}
\begin{claim}{}
The graph in $3.1.28$ does not have a perfect matching.
\end{claim}
\begin{proof}
By Tutte's theorem if there exists a $S \subseteq V(G)$ so that $o(G - S) > |S|$ then $G$ does not contain a perfect matching. Consider the attached drawing of the graph that fixes a $S$ such that the conditions above hold. As a result, $G$ does not contain a perfect matching.
\end{proof}

\end{document}
