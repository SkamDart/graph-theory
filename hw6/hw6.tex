% !TeX spellcheck = en_US

% ===============================================
% MATH 412: Graph Theory         Fall 2017
% hw_revised.tex




\documentclass{article}

\usepackage[margin=1in]{geometry} 
\usepackage{amsmath,amsthm,amssymb, enumitem, mathtools}

\newcommand{\R}{\mathbb{R}}  
\newcommand{\Z}{\mathbb{Z}}
\newcommand{\N}{\mathbb{N}}
\newcommand{\Q}{\mathbb{Q}}
\newcommand{\C}{\mathbb{C}}
\newcommand{\K}[2]{K_{\floor*{#1} \ceil*{#2}}}

\newenvironment{theorem}[2][Theorem]{\begin{trivlist}
\item[\hskip \labelsep {\bfseries #1}\hskip \labelsep {\bfseries #2.}]}{\end{trivlist}}
\newenvironment{lemma}[2][Lemma]{\begin{trivlist}
\item[\hskip \labelsep {\bfseries #1}\hskip \labelsep {\bfseries #2.}]}{\end{trivlist}}
\newenvironment{exercise}[2][Exercise]{\begin{trivlist}
\item[\hskip \labelsep {\bfseries #1}\hskip \labelsep {\bfseries #2.}]}{\end{trivlist}}
\newenvironment{problem}[2][Problem]{\begin{trivlist}
\item[\hskip \labelsep {\bfseries #1}\hskip \labelsep {\bfseries #2.}]}{\end{trivlist}}
\newenvironment{question}[2][Question]{\begin{trivlist}
\item[\hskip \labelsep {\bfseries #1}\hskip \labelsep {\bfseries #2.}]}{\end{trivlist}}
\newenvironment{corollary}[2][Corollary]{\begin{trivlist}
\item[\hskip \labelsep {\bfseries #1}\hskip \labelsep {\bfseries #2.}]}{\end{trivlist}}

\DeclarePairedDelimiter\ceil{\lceil}{\rceil}
\DeclarePairedDelimiter\floor{\lfloor}{\rfloor}

\newenvironment{solution}{\begin{proof}[Solution]}{\end{proof}}
\newenvironment{claim}[2][Claim]{\begin{trivlist}
		\item[\hskip \labelsep {\bfseries #1}\hskip \labelsep {\bfseries #2}]}{\end{trivlist}}

\begin{document}
\title{Homework 6} 
\author{Cameron Dart\\ Graph Theory} 

\maketitle

\noindent \textbf{Problem 1}
See attached sheet for Male and and Female perfect matchings\\ \\

\noindent \textbf{Problem 2}
\begin{claim}{}
	All $k-$regular bipartite graphs satisfy Tutte's Condition.\\
\end{claim}
\begin{proof}
Suppose $G$ is a $k-$regular bipartite graph. If $k = 1$, then our bipartite graph is a collection of disjoint paths. For any arbitrary $S \subseteq V(G)$ the number of odd components in $G$ is equal to the number of vertices taken from unique paths in $G$. We do not add any odd components to $G - S$ if we take both vertices contained in a path or leave paths. As a result, $o(G - S) \leq |S|$.
\begin{lemma}A Let $k \geq 2$. Suppose there are $u,v \in V(G)$ with cut edge $uv$ that lie in different components of $G - uv$.  Let $u \in A$ that is a bipartite subgraph of $G$ and $v \in B$. Every vertex in $A$ has degree $k$ besides $u$ that has degree $k - 1$. This is not possible so we reach a contradiction. Hence, there are no cut edges in any $k-$regular bipartite graphs where $k \geq 2$. 
\end{lemma}
Let $k \geq 2$ and $G$ be a connected $k-$regular bipartite graph. By \textit{Lemma A} there cannot be any cut edges in any connected components of $G$.\\
Note, all components in $G$ are $K_{k,k}$ and have $2k$ vertices.
Suppose $G$ consists of $n$ components where each component is a $k-$regular bipartite graph. Deleting any number of vertices from $G$ will  
%Suppose $S \subseteq G$. Since $G$ has no cut edges, removing any set of vertices will not introduce additional components to the resulting graph. Hence, for each component in $G$ we can create at most $1$ odd component. Hen
\end{proof}

\noindent \textbf{Problem 3}
\begin{claim}{}
Suppose $G$ is a $7-$regular connected graph that remains connected after deleting $5$ edges. Then $G$ has a perfect matching.
\end{claim}
\begin{proof}
Let $G'$ be the graph that results from deleting $e_1,e_2,e_3,e_4,e_5$ edges from $G$. Since $G'$ is still connected we know that $G$ did not contain any cut edges. So $o(G) \leq 1$.
\end{proof}

\noindent \textbf{Problem 4}
\begin{claim}{}
There exists a $5-$regular simple connected graph that remains connected after deleting any $2$ edges but does not have a perfect matching.
\end{claim}
\begin{proof}
See attached sheet for drawing.
\end{proof}

\noindent \textbf{Problem 5}
\begin{claim}{}

\end{claim}
\begin{proof}

\end{proof}

\end{document}
