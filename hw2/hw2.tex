% !TeX spellcheck = en_US

% ===============================================
% MATH 412: Graph Theory         Fall 2017
% hw_revised.tex
% Template for revised homework submission
% ===============================================
%         READ THE FOLLOWING CAREFULLY!!!
% ===============================================
% When you produce a PDF version of this document
% to turn in, change the filename to hwX-name.pdf
% replacing X with the homework assignment number
% and name with your last name.
% ===============================================


% -----------------------------------------------
% The preamble that follows can be ignored. Go on
% down to the section that says "START HERE" 
% -----------------------------------------------

\documentclass{article}

\usepackage[margin=1in]{geometry} 
\usepackage{amsmath,amsthm,amssymb, enumitem}

\newcommand{\R}{\mathbb{R}}  
\newcommand{\Z}{\mathbb{Z}}
\newcommand{\N}{\mathbb{N}}
\newcommand{\Q}{\mathbb{Q}}
\newcommand{\C}{\mathbb{C}}

\newenvironment{theorem}[2][Theorem]{\begin{trivlist}
\item[\hskip \labelsep {\bfseries #1}\hskip \labelsep {\bfseries #2.}]}{\end{trivlist}}
\newenvironment{lemma}[2][Lemma]{\begin{trivlist}
\item[\hskip \labelsep {\bfseries #1}\hskip \labelsep {\bfseries #2.}]}{\end{trivlist}}
\newenvironment{exercise}[2][Exercise]{\begin{trivlist}
\item[\hskip \labelsep {\bfseries #1}\hskip \labelsep {\bfseries #2.}]}{\end{trivlist}}
\newenvironment{problem}[2][Problem]{\begin{trivlist}
\item[\hskip \labelsep {\bfseries #1}\hskip \labelsep {\bfseries #2.}]}{\end{trivlist}}
\newenvironment{question}[2][Question]{\begin{trivlist}
\item[\hskip \labelsep {\bfseries #1}\hskip \labelsep {\bfseries #2.}]}{\end{trivlist}}
\newenvironment{corollary}[2][Corollary]{\begin{trivlist}
\item[\hskip \labelsep {\bfseries #1}\hskip \labelsep {\bfseries #2.}]}{\end{trivlist}}

\newenvironment{solution}{\begin{proof}[Solution]}{\end{proof}}
\newenvironment{claim}[2][Claim]{\begin{trivlist}
		\item[\hskip \labelsep {\bfseries #1}\hskip \labelsep {\bfseries #2}]}{\end{trivlist}}

\begin{document}
\title{Homework 2} % Change homework number
\author{Cameron Dart\\ Graph Theory} % Replace "Author's Name" with your name

\maketitle

%%%%%%%%%%%%%%%%%%%%%%%%%%%%%%%%%%%%%%%%
\noindent\textbf{Problem 1}
\begin{claim}{}
	A graph is bipartite iff every subgraph $H$ of $G$ has an independent set of size at least $\frac{|V(H)|}{2}$
\end{claim}

\begin{proof}
	$\implies $ Suppose $G$ is a bipartite graph. Then $G$ must not contain any odd cycles. We cannot introduce any cycles by removing any edges or vertices from a graph that does not initially contain any. Hence, any subgraph $H$ must also be bipartite and therefore split into two independent sets. It follows that one of the independent sets must be at least $\frac{|V(H)|}{2}$.\\ \\
	$\impliedby$ Consider the contrapositive, if $G$ is not bipartite then there exists a subgraph $H$ of $G$ such that there is not an independent set of size at least $\frac{|V(H)|}{2}$. If $G$ is not bipartite then there exists an odd cycle. Suppose $H$ is the subgraph of $G$ that is $C_n$ for some odd integer $n$. $H$ does not have an independent set that is at least $\frac{|V(H)|}{2}$.\\
	\end{proof}

\noindent\textbf{Problem 2}
\begin{claim}{a}
	Let $n \geq 4$ and $G$ be an $n-$vertex simple connected graph not containing $4-$vertex path $P_4$ as an induced subgraph 
	If $G$ is not complete bipartite, then $G$ contains the cycle $C_3$
\end{claim}

\begin{proof}
	Consider the contrapositive of our claim. If $G$ does not contain the cycle, then $G$ is complete bipartite. Since $G$ is connected and there are no $3$-cycles in $G$ there exists $u,v,w \in V(G)$ such that the subgraph $u,v,w$ is complete bipartite. Now consider $z \in V(G)$ that is connected to any of $u,v,w$. There must not be a 3-cycle between any three vertices of $u,v,w,z$. If $z$ is connected to one of our independent sets in $u,v,w$ then it must be connected to all elements in that independent set. Otherwise, we have an induced subgraph of $P_4$ which contradicts our original assumption. Hence, $G$ is complete bipartite.
\end{proof}

\begin{claim}{b}
	If $G$ has no vertex adjacent to all other vertices, then $G$ has the cycle $C_4$ as an induced subgraph
\end{claim}

\begin{proof}
	Suppose $G$ has no vertex that is adjacent to all others vertices. Let $u  = \Delta(G)$, $v$ and $w$ be any two vertices adjacent to $u$, and $z$ is a vertex not adjacent to $z$. In order for our graph to be connected, $z$ must connect to $v$ or $w$. 
	However, if it only connects to one, then we have an induced subgraph of $P_4$. So it must connect to both. Hence, $u,v,w,z$ contains a 4-cycle.\\
\end{proof}

\noindent\textbf{Problem 3}
\begin{claim}{}
	If $G$ is a $5-$regular graph, then $E(G)$ cannot be partitioned in paths of length at least $7$.
\end{claim}
\begin{proof}
 $G$ Suppose $G$ is a $n$ vertex 5-regular graph. Assume that $E(G)$ can be partitioned into paths of length at least $7$. $G$ has $\frac{5n}{2}$ edges. Note, this means $n$ must be even. Since paths must start and end at different vertices, we need $\frac{n}{2}$ paths to cover $G$. So if our paths need to be at least length $7$ we need a minimum $\frac{7n}{2}$ edges. Since $\frac{7n}{2} > \frac{5n}{2}$ we reach out sought contradiction.
\end{proof}
\newpage
\noindent\textbf{Problem 4}
\begin{claim}{}
	Every cycle of length $2r$ in the hypercube $Q_n$ is contained in a subcube of dimension at most $r$
\end{claim}
\begin{proof}
	Consider a hypercube $Q_n$ and a cycle of length $2r$. In a hypercube, $u,v$ are adjacent if their coordinates differ by 1. So if two vertices share a common neighbor, their coordinates differ by at most $2$, and so on. So a path of length $r$ differs in at most $r$ coordinates. Hence, our path of length $r$ spans at most $r$ dimensions and our cycle is contained in some subcube of dimension at most $r$.
\end{proof}
Additionally, yes a cycle of length 8 can be contained in a subcube of dimension $2$ and
a cycle of length $6$ cannot be contained in $2$ dimensions.\\

\noindent\textbf{Problem 5}
\begin{claim}{5}
	$G$ has $6$ equivalence classes.
\end{claim}
\begin{proof}
Initially there are $4!$ permutations of our $4$ paths, however, we don't distinguish between the starting vertex $x,y$ and the reverse cycles are equivalent so we are overcounting by a factor of $4$. Hence, there are $6$ equivalence classes.
\end{proof}
\end{document}